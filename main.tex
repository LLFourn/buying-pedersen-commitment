\documentclass{article}
\addtolength{\oddsidemargin}{-.875in}
\addtolength{\evensidemargin}{-.875in}
\addtolength{\textwidth}{1.75in}
\addtolength{\topmargin}{-.875in}
\addtolength{\textheight}{1.75in}

\usepackage{mdframed}
\usepackage[utf8]{inputenc}
\usepackage[n,advantage,operators,sets,adversary,landau,probability,notions,logic,ff, mm, primitives,events,complexity,asymptotics,keys]{cryptocode}
\usepackage{caption}

\title{Note on Paying for a Pedersen Commitment Opening}
\author{Lloyd Fournier (lloyd.fourn@gmail.com)}
\date{December 2019}

\newcommand{\Pdleq}{\pcalgostyle{P}_{\textsf{DLEQ}}}
\newcommand{\Vdleq}{\pcalgostyle{V}_{\textsf{DLEQ}}}
\captionsetup[figure]{labelfont={bf},name={Fig.},labelsep=period}


\begin{document}

\maketitle

\section{Creating The Access Structure}

In this short note we demonstrate how a to securely sell an opening $(r,x)$ to a Pedersen Commitment $C = rG + xH$ on Bitcoin. The protocol begins with the seller publishing a simple discrete log access structure for the opening. If the buyer is able to learn the discrete logarithms of two points ($A$ and $B$ below) the they will be able to decrypt the opening.

\begin{figure}[h]
  \centering
  \begin{mdframed}
    \begin{center}
 \pseudocode{
   \textbf{Seller}(r,x) \< \< \textbf{Buyer} \\[0.1 \baselineskip][\hline]\\
     (a,b) \sample \ZZ_q \\
     A \gets aG; B \gets bG; B' \gets bH \\
     \alpha \gets r + a; \beta \gets x + b \\
     \pi \gets \Pdleq((G,B),(H,B'), b) \\
     \< \sendmessageright*{\alpha, \beta, A, B, B', \pi} \< \\
     \< \< \Vdleq((G,B), (H,B'), \pi) \stackrel{?}{=} 1\\
     \< \< C + A + B' \stackrel{?}{=} \alpha G + \beta H \\
     \pcreturn ((a,A),(b,B)) \< \< \pcreturn (A,B, \alpha, \beta)
   }
   \end{center}
 \end{mdframed}
 \caption{The access structure setup protocol. $(\Pdleq, \Vdleq)$ denote the non-interactive proving and verification algorithms for discrete logarithm equality \cite{chaum1983blind}}
\end{figure}

\subsection{Security}

Correctness follows from the fact that $C = rG + xH  = \alpha G + \beta H - A - B'$. If the buyer learns $a$ and $b$ such that $A = aG$ and $B = bG$, then $(r = \alpha - a, x = \beta - b)$ must be a valid opening for $C$ (if the proof $\pi$ is sound). We must also ensure that the buyer learns nothing about the opening until they learn $a$ and $b$. This follows from the fact that a valid looking tuple $(\alpha, \beta, A, B, B', \pi)$ is \emph{simulatable} i.e. it can be produced without knowledge of the opening of $C$: choose $b, \alpha, \beta$ as normal and then set $A \gets \alpha{}G + \beta{}H - B' - C$.

\section{Purchasing the dicrete logaritihms}

\newcommand{\I}{\mathbf{I}}
\renewcommand{\i}{\mathbf{i}}
\newcommand{\addr}{\texttt{addr}}
\newcommand{\Tx}{\textsf{Tx}}
\newcommand{\Fund}{\textsf{Fund}}
\newcommand{\Refund}{\textsf{Refund}}
\newcommand{\Redeem}{\textsf{Redeem}}

The buyer with $(A,B,\alpha,\beta)$, can generate a valid opening for $C$ given ($a$,$b$), the discrete logarithms of $(A,B)$. Thus, the buyer now attempts to purchase $(a,b)$ from the seller. In case, the seller does not know $(a,b)$ the buyer must have their money returned. Thus they construct the following transaction scaffold

\begin{enumerate}
\item \Fund: Spends from the buyers inputs with two outputs whose value adds up to $v$.
\item \Redeem: Spends the two outputs of $\Fund$ to the seller's address
\item \Refund: Spends the two outputs of $\Fund$ to the buyer's address (time-locked)
\end{enumerate}

To complete the scaffold they define transitions between transactions in the scaffold by exchanging signatures on the $\Redeem$ and $\Refund$ transactions as follows:

\begin{enumerate}
\item They jointly sign both inputs of the $\Refund$ transaction such that the buyer has a valid witness for it.
\item They jointly produce two one-time encrypted signatures on the inputs for $\Redeem$ encrypted by $A$ and $B$ respectively.
\end{enumerate}

This completes the scaffold. If the buyer wants to go through with the purchase they sign and broadcast the $\Fund$ transaction. Then, if the seller wants to go through with the sale they decrypt both one-time encrypted signatures on the $\Redeem$'s inputs with $(a,b)$ and broadcast $\Redeem$. From the one-timeness of the encryptions the buyer learns ($a$,$b$) and can therefore recover $r \gets \alpha - a$ and $x \gets \beta - b$.

\bibliography{bib.bib}{}
\bibliographystyle{unsrt}

\end{document}
